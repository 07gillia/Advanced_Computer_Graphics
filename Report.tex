%\title{Project Report}
%
%%% Preamble
\documentclass[paper=a4, fontsize=11pt]{scrartcl}
\usepackage[T1]{fontenc}
\usepackage{fourier}
\usepackage{listings}
\usepackage{times}
\usepackage{harvard}

\usepackage[english]{babel}															% English language/hyphenation
\usepackage[protrusion=true,expansion=true]{microtype}	
\usepackage{amsmath,amsfonts,amsthm} % Math packages
\usepackage[pdftex]{graphicx}	
\usepackage{url}
\usepackage{hyperref}
\usepackage{graphicx}
\usepackage{wrapfig}
\usepackage[margin=1.00in]{geometry}
\usepackage{amsmath}
\usepackage[]{algorithm2e}

\usepackage{amsmath}
\usepackage{amssymb}
\usepackage{amsthm}
\usepackage{amsfonts}
\usepackage{braket}

\citationmode{abbr}
\bibliographystyle{agsm}

\DeclareOldFontCommand{\rm}{\normalfont\rmfamily}{\mathrm}
\DeclareOldFontCommand{\sf}{\normalfont\sffamily}{\mathsf}
\DeclareOldFontCommand{\tt}{\normalfont\ttfamily}{\mathtt}
\DeclareOldFontCommand{\bf}{\normalfont\bfseries}{\mathbf}
\DeclareOldFontCommand{\it}{\normalfont\itshape}{\mathit}
\DeclareOldFontCommand{\sl}{\normalfont\slshape}{\@nomath\sl}
\DeclareOldFontCommand{\sc}{\normalfont\scshape}{\@nomath\sc}
\DeclareRobustCommand*\cal{\@fontswitch\relax\mathcal}
\DeclareRobustCommand*\mit{\@fontswitch\relax\mathnormal}

%%% Custom sectioning
\usepackage{sectsty}
\allsectionsfont{\centering \normalfont\scshape}


%%% Custom headers/footers (fancyhdr package)
\usepackage{fancyhdr}
\pagestyle{fancyplain}
\fancyhead{}											% No page header
\fancyfoot[L]{}											% Empty 
\fancyfoot[C]{}											% Empty
\fancyfoot[R]{\thepage}									% Pagenumbering
\renewcommand{\headrulewidth}{0pt}			% Remove header underlines
\renewcommand{\footrulewidth}{0pt}				% Remove footer underlines
\setlength{\headheight}{3.6pt}
\date{}


%%% Equation and float numbering
\numberwithin{equation}{section}		% Equationnumbering: section.eq#
\numberwithin{figure}{section}			% Figurenumbering: section.fig#
\numberwithin{table}{section}				% Tablenumbering: section.tab#


%%% Maketitle metadata
\newcommand{\horrule}[1]{\rule{\linewidth}{#1}} 	% Horizontal rule

\title{
		\vspace{-1in} 	
		\usefont{OT1}{bch}{b}{n}
		\normalfont \normalsize \textsc{Durham Computer Science} \\ [5pt]
		\horrule{0.5pt} \\[0.4cm]
		\huge  Advanced Computer Graphics Summative Assignment - LLLL76\\
		\horrule{2pt} \\[0.5cm]
		\vspace{-1in} 	
}

%%% Begin document
\begin{document}
\maketitle

\section*{Question One - 20 marks}

Compare the main difference between applying appearance-based metric and geometric based
metric to measure the quality difference between two polygon meshes. Analyse in
which part of the graphics rendering pipeline each metric should be applied to perform quality
measurement. \\



\section*{Question Two - 10 marks}

Explain how the Hausdorff distance can serve as a metric to determine the dissimilarity
between two polygon meshes, even when these meshes are formed by different number
vertices and connectivity.\\



\section*{Question Three - 20 marks}

Describe the data structure of progressive meshes. Analyse the rendering efficiency of
progressive meshes visualisation, given that the user is allowed to freely rotate the viewpoint
during the visualisation process.\\



\section*{Question Four - 10 marks}

Explain how progressive meshes implement the refinement and decimation processes.\\



\section*{Question Five - 20 marks}

 Analyse how the incorporation of level-of-detail modeling impacts the rendering performance
and network bandwidth consumption of a large distributed virtual environment system.
Evaluate the suitability of using progressive meshes to implement the level-of-detail modeling
in such a system.\\
Note that in the above distributed virtual environment system, all graphics models of the virtual
environment are maintained by a remote server. During runtime, each client will download
relevant graphics models on-demand from the server to support interaction and visualisation.



\section*{Question Six - 20 marks}

Explain the main issue of applying level-of-detail modeling to support interactive visualisation
of a large 3D scene, given that the user is allowed to change the viewpoint from time to time.
Describe two different methods to tackle such a challenge. \\



\end{document}